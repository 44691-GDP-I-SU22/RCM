
\section*{Introduction} Different content management systems have been studies, such as, learning management system~\cite{abazi2008development}, content management system~\cite{mooney2008extensible,michelinakis2004open}, course management system~\cite{brooks2006awareness}, Web-based content management system~\cite{dobecki2010web}. In addition, Researchers uses reference management tools~\cite{singh2010mendeley,eapen2006endnote,grolimund2012citation,beel2011docear} for maintaining the history of research papers read by them. Most of available reference website/tools provides researchers to create  bibliographies and citations. Some of them also allow researchers to collaborate with others. Using these tools, users can store, organize and search all available references.
 However, it is hard for the researchers to remember what they have read in the past months and years.
Hence, we focus on developing an application that could assist researchers to read papers online and allow them to write their notes about the paper. Although the application is beneficial to all the researchers, it will be more useful to multidisciplinary investigators. 

In the application, we allows users to upload their interested papers with topic, year, title, and techniques. This helps the researchers to find all their interested papers based on research topic, year, title, and techniques. For each uploaded paper, the application provides an interactive window for the users to read the paper, write comments on the paper. The comments are stored along with time. 

The application consists of two parts, namely, front and back end. For the front-end part, we have developed interactive web pages for the users to register, login, upload and search papers based on one or more criteria, and view a selected paper and write notes/comments about the paper. All the pages have been developed using React. For the back-end part, we have used Firebase to store the uploaded papers and comments. 
